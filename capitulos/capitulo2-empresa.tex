\chapter{A Empresa}
\label{cap2}

A empresa que me hospedou num estágio foi a \href{https://optiply.nl/}{Optiply}, durante três meses (do segundo semestre do ano letivo de 2021/2022), para uma posição de aprendizagem de desenvolvimento de backend.

\section{Caracterização}

A \href{https://optiply.nl/}{Optiply} é uma empresa de software cujo produto/serviço é a gestão inteligente de \textit{stocks} dos produtos e serviços de lojas online. Esta empresa foi fundada em 2015 em Amesterdão, Países Baixos e que depressa (em 2017), expandiu a sua equipa de desenvolvido de software para Évora, Portugal.

Em Évora, esta empresa detém por volta de 20 empregados, dos quais três quartos são engenheiros de software, distribuídos em frontend, backend e integrações.

\section{Organização e Comunicação}

Apesar da empresa deter instalações e equipamentos, esta suporta trabalho remoto, o qual me pareceu ser a escolha da maioria dos empregados que estão em desenvolvimento de software, eu incluso. O qual para a Comunicação da empresa, inter e entre empregados era essencialmente via Slack, com canais de comunicação gerais, divididos em equipas, e comunicação privada. Evitava-se ao máximo o uso de video chamadas, e ao invés de isso, usava-se mensagem de texto, à exceção do \textit{Tech Lead/Team Leader} que usava mensagem de texto e video chamada para coordenar as equipas locais e estrangeiras.

Para a gestão dos projetos, esta é uma empresa \href{https://www.atlassian.com/agile}{\textit{Agile}}, o qual obviamente usa a suite da \href{https://www.atlassian.com/}{Atlassian}, nomeadamente o \href{https://bitbucket.org/}{BitBucket} para hospedar os repositórios de software, bibliotecas e frameworks, como também, o \href{https://jira.atlassian.com/}{Jira}.

\section{Tech Stack}

\lipsum

\section{Produto}

\lipsum
