\chapter{A Empresa}
\label{cap2}

\begin{figure}[!hbt]
  \centering
  \includegraphics[width=12cm]{figuras/optiply_logo.png}
  \caption{Logo da empresa}
  \label{fig:logo}
\end{figure}
\FloatBarrier

A empresa que me hospedou num estágio foi a \href{https://optiply.nl/}{Optiply}, durante três meses, para uma posição de aprendizagem de desenvolvimento de backend.

\section{Caracterização}

A \href{https://optiply.nl/}{Optiply} é uma empresa de software cujo produto/serviço é a gestão inteligente de \textit{stocks} dos produtos e serviços de lojas online. Esta empresa foi fundada em 2015 em Amesterdão, Países Baixos e que depressa (em 2017), expandiu a sua equipa de desenvolvido de software para Évora, Portugal.

Em Évora, esta empresa detém por volta de 20 empregados, dos quais três quartos são engenheiros de software, distribuídos em frontend, backend e integrações.

\section{Produto}

Como referido brevemente na secção anterior esta é uma empresa de SaaS, ou seja um software vendido como um serviço, o uso deste software está associado a uma subscrição que oferece os serviços promovidos. Mais especificamente são serviços de inteligência artificial que fazem sugestões automáticas de compras e promoções para o cliente, facilitando a gestão dos stocks do seu armazém.

\newpage

\section{Organização e Comunicação}

Apesar da empresa deter instalações e equipamentos, esta suporta trabalho remoto, o qual me pareceu ser a escolha da maioria dos empregados que estão em desenvolvimento de software, eu incluso. O qual para a Comunicação da empresa, inter e entre empregados era essencialmente via Slack, com canais de comunicação gerais, divididos em equipas, e comunicação privada. Evitava-se ao máximo o uso de video chamadas, e ao invés de isso, usava-se mensagem de texto, à exceção do \textit{Tech Lead/Team Leader} que usava mensagem de texto e video chamada para coordenar as equipas locais e estrangeiras.

Para a gestão dos projetos, esta é uma empresa \href{https://www.atlassian.com/agile}{\textit{Agile}}, o qual obviamente usa a suite da \href{https://www.atlassian.com/}{Atlassian}, nomeadamente o \href{https://bitbucket.org/}{BitBucket} para hospedar os repositórios de software, bibliotecas e frameworks, como também, o \href{https://jira.atlassian.com/}{Jira}.

\section{\textit{Tech Stack}}

O serviço oferecido pela empresa foi desenvolvido por um conjunto de tecnologias, dos quais do tempo em que estive ativo consegui identificar que se usa a framework de Javascript: \href{https://angular.io/}{Angular} para a construção do front-end o qual pode ser acedido num browser ou empacotado numa aplicação \href{https://www.electronjs.org/}{Electron}, a qual comunica com um servidor, possivelmente Linux e \textit{\href{https://www.debian.org/}{Debian}-based}. Este servidor detêm variados containers \href{https://www.docker.com/}{Docker}, os quais são a infraestrutura vital do backend do serviço, hospedando as bases de dados \href{https://www.postgresql.org/}{PostgreSQL} e \href{https://www.mongodb.com/}{MongoDB}, e os \textit{microservices}, implementados nas frameworks de \href{https://jdk.java.net/}{Java}: \href{https://micronaut.io/}{Micronaut} (as mais recentes) e \href{https://spring.io/}{Spring}.

\begin{figure}[!hbt]
  \centering
  \includegraphics[width=12cm]{figuras/lol.png}
  \caption{Diagrama exemplar do serviço}
  \label{fig:diag_eg}
\end{figure}
\FloatBarrier