\chapter{Introdução}
\label{intro}

Este presente relatório tem como objetivo apresentar o decorrer do estágio profissional e as consequências ou resultados do mesmo, o qual ocorreu no período de 02/03/2022 a 02/06/2022, na cidade de Évora, Portugal. O estágio foi hospedado pela \href{https://optiply.nl/}{Optiply}, que é uma empresa de gestão inteligente de \textit{stocks} dos produtos e serviços de lojas online, cujo orientador (Fábio Belga) é o \textit{Team Leader/Tech Lead}, que gere todo o processo de desenvolvimento e gestão do projeto.

O meu papel como estagiário foi um de treino para desenvolvimento em backend com um pequeno projeto, um \textit{microservice} que realiza a gestão de especificações de lojas online. Este projeto envolveu variadas tecnologias e paradigmas de trabalho e de programação, os quais passam por diversas etapas de desenvolvimento, testes e documentação, mas no que toca à gestão e organização de projeto foi de escolha livre, ou seja, eu geria o meu tempo e o projeto à minha vontade sem vigilância ou controlo. O qual, admitindo a verdade, não geri o meu tempo de qualquer forma, apenas os objetivos de projeto em sí, num estilo primitivo de Kanban.

Com a leitura deste relatório, pretendo que gradualmente se expanda e detalhe o referido no paragrafo anterior, e que seja possível compreender o que foi aprendido e o que foi desenvolvido.

Na realização deste estágio, foram obtidos diversos conhecimentos que serão fundamentais para minha carreira profissional. Espero que este relatório possa contribuir para uma melhor avaliação do estágio e auxiliar na tomada de decisões futuras.