\chapter{Projeto desenvolvido}\label{cap5}

Este capitulo descreve em detalhe o funcionamento do projeto desenvolvido durante o decorrer do estágio.

\section{Descrição geral}

O \textit{Webshop Service Specification} é uma RESTful API reactiva, que consiste em gerir as especificações de Webshops, ou seja um microserviço reactivo. Este microserviço recebe pedidos HTTP e retorna uma resposta JSON com o resultado, nomeadamente uma Webshop ou uma caraterística da mesma.

Sendo esta API reactiva, ela emprega o uso de \textit{threading} (divisão de tarefas em subprocessos) para poder executar variados pedidos em simultâneo, o mais depressa possível. No entanto esses processos concorrentes e assíncronos requerem um outro nível de cuidado e atenção no que toca à integridade e idempotência dos dados requeridos.

As relações empregues por uma aplicação reactiva são os padrões de \textit{Publisher/Subscriber}, onde um pedido, ou uma transação, é uma mensagem enviada pela fonte desses dados, chamada de um \textit{Publisher} e a sua receção, ou seja onde os dados são consumidos, é encarregado pelo(s) \textit{Subscriber(s)}. Se um desses pedidos for uma mensagem com varias subscrições ao longo do tempo, devemos alterar o \textit{scheduling}, que gere as filas de processos e acessos.

Com isto podemos dizer que os \textit{endpoints} desta API são \textit{Subscribers} e o serviço transacional que comunica diretamente com os dados da base de dados é o nosso \textit{Publisher}. Este tipo de acessos reactivos à base dados requer um outro tipo de mecanismo de processo de transações SQL, o qual deve ser também reactivo de modo a que a base de dados seja vista e funcione com um \textit{Publisher} e seja configurável a sua propagação.

\newpage

\section{Organização}

Este projeto (como anteriormente mencionado), é um projeto Gradle que contem dois subprojetos, os quais gerem tarefas diferentes mas co-dependentes:

\begin{itemize}
  \item O pacote dos \hyperref[endp]{\textit{Endpoints}}: um pacote para o core do projeto, este contem a estrutura MC do projeto, com os respectivos modelos, controller e o serviço que comunica com os repositórios do pacote seguinte;
  \item O pacote da \hyperref[infra]{\textit{Infrastructure}}: pacote para os repositórios que tratam das transações com a base de dados e as classes geradas do jOOQ que os repositórios utilizam.
\end{itemize}

Dentro da \textit{root directory} do projeto contemos variadas subdirectorias e ficheiros, dos quais podemos apontar:

\begin{itemize}
  \item \texttt{\textbf{build/ -> }} diretoria onde o Gradle gera os binários, os \textit{jars}, artefactos, etc\ldots da tarefa de compilação;
  \item \texttt{\textbf{endpoints/ -> }} \textit{source directory} do subprojeto \hyperref[endp]{\textit{Endpoints}};
  \item \texttt{\textbf{gradle/ -> }} diretoria onde existem os \textit{wrappers} do Gradle;
  \item \texttt{\textbf{infrastructure/ -> }} \textit{source directory} do subprojeto \hyperref[infra]{\textit{Infrastructure}};
  \item \texttt{\textbf{javadoc/ -> }} diretoria onde se gera o JavaDoc, ou seja um documento/\textit{website} com a documentação (derivada dos \textit{block comments});
  \item \texttt{\textbf{build.gradle -> }} ficheiro de configuração \textbf{principal} do Gradle, onde definimos os pacotes a ir buscar e programamamos as tarefas de (pré e pós) compilação e de testes;
  \item \texttt{\textbf{gradle.properties -> }} ficheiro de configuração \textbf{opcional} do Gradle onde se definem \textit{compiler flags}, argumentos para a JVM e outras configurações mais profundas e especificas;
  \item \texttt{\textbf{lombok.config -> }} ficheiro de configuração \textbf{opcional} do Lombok, onde aqui defino para adicionar a anotação \texttt{@Generated} as suas classes geradas para \textit{fugir} ao JaCoCo;
  \item \texttt{\textbf{micronaut-cli.yml -> }} ficheiro de preferências da criação de um projeto Micronaut via a sua ferramenta CLI;
  \item \texttt{\textbf{postgres-compose.yml -> }} ficheiro de \textbf{docker-compose} para compor e lançar containers com pré configurações, neste caso um container de PostgreSQL;
  \item \texttt{\textbf{settings.gradle -> }} ficheiro de configuração do Gradle onde se definem os subprojetos do projeto Gradle, é executado a cada \textit{build task}.
\end{itemize}

\newpage

\subsection{\textit{\textit{Endpoints} package}}\label{endp}

Este é o subprojeto \textit{Endpoints} onde contêm toda a estrutura base da API, deste o ponto inicial da aplicação, às configurações, modelos, controladores e serviços. Tendo em conta que \texttt{com/optiply/endpoint/} fica como reticências, temos que:

\begin{itemize}
  \item \texttt{\textbf{src/main/}}\begin{itemize}
          \item \texttt{\textbf{java/}}\begin{itemize}
                  \item \texttt{\textbf{\ldots/config/}}\begin{itemize}
                    \item \texttt{\textbf{DataSourceConfig.java}}
                  \end{itemize}
                  \item \texttt{\textbf{\ldots/controllers/}}\begin{itemize}
                    \item \texttt{\textbf{shared/interfaces/IBaseController.java}}
                    \item \texttt{\textbf{shared/BaseController.java}}
                    \item \texttt{\textbf{JSONController.java}}
                  \end{itemize}
                  \item \texttt{\textbf{\ldots/models/}}\begin{itemize}
                    \item \texttt{\textbf{EmailListModel.java}}
                    \item \texttt{\textbf{HandleModel.java}}
                    \item \texttt{\textbf{InterestRateModel.java}}
                    \item \texttt{\textbf{ServiceLevelsModel.java}}
                    \item \texttt{\textbf{SettingsModel.java}}
                    \item \texttt{\textbf{UrlModel.java}}
                    \item \texttt{\textbf{WebshopFullModel.java}}
                    \item \texttt{\textbf{WebshopModel.java}}
                    \item \texttt{\textbf{WebshopSettingsModel.java}}
                  \end{itemize}
                  \item \texttt{\textbf{\ldots/services/}}\begin{itemize}
                    \item \texttt{\textbf{RepositoryService.java}}
                  \end{itemize}
                  \item \texttt{\textbf{\ldots/EndpointApplication.java}}
                \end{itemize}
          \item \texttt{\textbf{resources/}}\begin{itemize}
                  \item \texttt{\textbf{db/migration/}}\begin{itemize}
                          \item \texttt{\textbf{V1\_\_create\_initial\_schema.sql}}
                        \end{itemize}
                  \item \texttt{\textbf{application.yaml}}
                  \item \texttt{\textbf{bootstrap.yaml}}
                  \item \texttt{\textbf{logback.xml}}
                \end{itemize}
        \end{itemize}
\end{itemize}

\newpage

\begin{itemize}
  \item \texttt{\textbf{src/test/}}\begin{itemize}
          \item \texttt{\textbf{java/}}\begin{itemize}
                  \item \texttt{\textbf{integration/}}\begin{itemize}
                          \item \texttt{\textbf{\ldots/controllers/JSONControllerIntegrationTests.java}}
                        \end{itemize}
                  \item \texttt{\textbf{shared/}}\begin{itemize}
                          \item \texttt{\textbf{\ldots/container/TestContainer.java}}
                          \item \texttt{\textbf{\ldots/environment/TestEnvironment.java}}
                        \end{itemize}
                  \item \texttt{\textbf{unit/}}\begin{itemize}
                          \item \texttt{\textbf{\ldots/models/WebshopFullModelsUnitTests.java}}
                          \item \texttt{\textbf{\ldots/services/RepositoryServiceUnitTests.java}}
                        \end{itemize}
                \end{itemize}
          \item \texttt{\textbf{resources/}}\begin{itemize}
                  \item \texttt{\textbf{application-test.yaml}}
                \end{itemize}
        \end{itemize}
\end{itemize}

Sendo que é clara a funcionalidade de cada classe pelo nome e pelo local onde se encontra. Mesmo sendo esse o caso, seguimos para uma explicação breve do que cada classe ou ficheiro faz, excluindo as classes de modelos, pois são obviamente modelos dos objetos transacionais (os corpos em JSON do \textit{request} HTTP) ou de suas partes.

\subsubsection*{\texttt{DataSourceConfig.java}}

Esta classe executa o carregamento e pós-configuração das configurações do ficheiro de configuração \texttt{application.yaml}, criando o contexto DSL (uma interface de comunicação do jOOQ com a base de dados via JDBC), e consequentemente cria a \textit{ConnectionFactory}, que permite usar R2DBC e fazer queries transacionais de forma reactiva.

\subsubsection*{\texttt{JSONController.java}}

Esta classe é um controlador de \textit{requests} HTTP com \textit{payloads} em JSON. Esta, extende a classe de controlador base \texttt{BaseController.java} (abstrata), que por si é uma implementação da interface \texttt{IBaseController.java}, que contem as funções der \textit{parsing} dos parâmetros de procura e sorteamento do \textit{endpoint} de pesquisa de Webshops.

\subsubsection*{\texttt{RepositoryService.java}}

Esta classe é responsável por deter toda a \textit{business logic} necessária e acessivel pelos \textit{controllers} e que os isola de contacto com os repositórios de dados. É aqui que se executam as tarefas que queremos executadas e recebemos os resultados quando usamos os \textit{endpoints} do \textit{controller}.

\newpage

\subsubsection*{\texttt{EndpointApplication.java}}

Classe principal/base de onde executa a aplicação.

\subsubsection*{\texttt{TestContainer.java} \textsuperscript{\textit{Testes}}}

\subsubsection*{\texttt{TestEnvironment.java} \textsuperscript{\textit{Testes}}}

\subsubsection*{\texttt{JSONControllerIntegrationTests.java} \textsuperscript{\textit{Testes}}}

\subsubsection*{\texttt{WebshopFullModelsUnitTests.java} \textsuperscript{\textit{Testes}}}

\subsubsection*{\texttt{RepositoryServiceUnitTests.java} \textsuperscript{\textit{Testes}}}

\subsubsection*{\texttt{V1\_\_create\_initial\_schema.sql}}

Ficheiro \texttt{.sql} com o esquema inicial da base de dados, onde o Gradle executa uma tarefa com o pacote FlyWay para fazer a migração. Esta base de dados contêm duas tabelas, uma com Webshops e as suas caraterísticas e outra para os emails de contacto das Webshops, relação de um-para-muitos.



\newpage

\subsection{\textit{\textit{Infrastructure} package}}\label{infra}

\section{Funcionamento}

\subsection{\textit{JSON Controller}}

\subsection{\textit{Webshop Service}}

\subsection{Modelos para comunicação}

\subsection{Modelos Internos}

\subsection{Repositórios}

\subsubsection{\textit{WebshopRepository}}

\subsubsection{\textit{WebshopemailsRepository}}
