\chapter{Desenvolvimento do projeto}\label{cap4}

Este capítulo descreve o projeto atribuído no estágio e o seu desenvolvimento. Sendo que este capítulo será o mais longo, mas consequentemente o mais importante e o mais complexo.

\section{Introdução}

Como anteriormente referido foi me destacada a tarefa de Implementação de um projeto no estágio. Este projeto consiste em um software que permite a gestão das especificações de Webshops.

Estas Webshops são, como o nome indica, as lojas online as quais são clientes da Optiply. Estas lojas online são responsáveis por fornecer os produtos que os clientes compram e a Optiply é responsável por fornecer a gestão inteligente dos produtos em stock.

O trabalho foi recebido num \textit{.pdf}, numa reunião de video-conferencia, com o coordenador do estágio (Fábio Belga), o seu subordinado (André Figueira)que ficou encarregado de orientar os estagiários de Backend, e nós (eu, Gonçalo Amaro e o estagiário da universade de Évora, José Azevedo), após a nossa fase formativa do \textit{Onboarding} descrita no capítulo anterior.

Assim, as primeiras secções deste capítulo servem como uma apresentação do equivalente à minha introdução ao projeto.

\section{Objetivos}

O objetivo descrito deste projeto é desenvolver um microserviço que permita a gestão das especificações de Webshops, já o verdadeiro objetivo deste projeto é fornecer treino ao estagiário nas tecnologias da \textit{Tech Stack} da empresa, ou pelo menos num dos projetos da mesma.

\newpage

Essa \textit{Tech Stack} referida é a seguinte:

\begin{itemize}
  \item Micronaut: Framework de desenvolvimento de microserviços.
  \item Java: Linguagem de programação.
  \item Gradle: Sistema de gestão de dependências e tarefas.
  \item jOOQ: Framework de código-fonte para acesso a bases de dados.
  \item Flyway: Framework de migração de bases de dados.
  \item PostgreSQL: Sistema de bases de dados.
  \item Junit5 (Spock também é aceitável): Framework de testes.
  \item Mockito: Framework de auxiliar a testes via simulação.
\end{itemize}

Voltando ao objetivo escrito do projeto (desenvolver um microserviço que permita a gestão das especificações de Webshops), o objetivo é desenvolver uma RESTful API que permita gerir as especificações de Webshops.

Para isso temos de saber que cada Webshop têm um conjunto de especificações, as quais são:

\begin{itemize}
  \item \textit{URL:} URL da loja online, têm validação e requer protocolo na URL;
  \item \textit{Handle:} identificador único da loja online;
  \item \textit{Interest Rate:} taxa de juros que a loja online paga, 20\% é o valor por defeito;
  \item \textit{Service Level Categories:} categorias de níveis de serviço que a loja têm, são três categorias (A,B e C) e as suas somas requerem ser iguais a 100\%;
  \item \textit{Contact Email List:} lista de emails de contacto da loja online, têm validação;
  \item \textbf{Extra:} \textit{Settings:} configurações da loja online:
    \begin{itemize}
      \item \textit{Enable Multi Supplier:} permite múltiplos fornecedores;
      \item \textit{Enable Run Jobs:} permite execução de tarefas;
      \item \textit{Currency:} moeda da loja online em ISO-4217;
    \end{itemize}
\end{itemize}

Sendo que as ultimas especificações (as \textit{Settings}) são Extras, ou seja, não são obrigatórias, mas foram implementadas.
\newpage

Essa API tem um determinado conjunto de tarefas a cumprir as quais são:

\begin{itemize}
  \item Obter uma única Webshop;
  \item Obter várias Webshops:
    \begin{itemize}
      \item Deve ser capaz de ordenar e filtrar por qualquer campo da tabela;
      \item Só é necessário ordenar por um único campo. Os resultados devem ser consistentes com cada pedido. (Se ordenar por Taxa de Juros, como pode-se garantir que os mesmos resultados sejam obtidos em todos os pedidos?)
      \item Só é necessário filtrar por um único campo. Os filtros suportados são:
        \begin{itemize}
          \item ``:'' significa \textit{Igual}. Exemplo: handle:optiply
          \item ``\%'' significa \textit{ILIKE} (semelhante, \textit{case-insensitive}).\\Exemplo: handle\%optiply
          \item \textbf{Extra:} ``>'' significa \textit{Maior Que}. Exemplo: interestRate>20
          \item \textbf{Extra:} ``<'' significa \textit{Menor Que}. Exemplo: interestRate<20
        \end{itemize}
    \end{itemize}
  \item Apagar uma única Webshop.
  \item Criar uma única Webshop.
  \item Atualizar qualquer campo da Webshop.
  \item \textbf{Extra:} Filtrar por múltiplos campos.
  \item \textbf{Extra:} Criar múltiplas Webshops.
  \item \textbf{Extra:} Obter as configurações da Webshop.
  \item \textbf{Extra:} Atualizar as configurações da Webshop.
\end{itemize}

Tendo sempre em conta que os resultados devem ser idempotentes e no seu estado mais recente e que os pedidos HTTP retornam:

\begin{itemize}
  \item Criar deve retornar 201.
  \item Obter e Atualizar devem retornar 200.
  \item Apagar deve retornar 204.
  \item Qualquer pedido deve retornar 404 se a loja não existir.
  \item Qualquer outro erro interno deve retornar 500 (Erro Interno).
\end{itemize}

Esta lista (tradução do que está no \textit{.pdf} recebido, que está também no \hyperref[ap1]{Apêndice I}), é bastante extensa, mas é bastante simples para entender o que é.

No entanto é estupidamente obscura a segunda intenção da lista, esta era a lista implícita de \textit{endpoints} da API\@.

O qual inicialmente não vendo uma lista de \textit{endpoints} explicita nem uma mera referência na reunião, a primeira iteração do trabalho usei os \textit{endpoints} que eu achava mais convenientes para o trabalho. Escusado será dizer, que tive de os refazer após a primeira receção de \textit{feedback}.

\newpage

\section{Implementação}

\subsection{Pré-Requisitos}

Para começar a implementar a API, precisamos de um conjunto de ferramentas. Essas ferramentas passam por um JDK (um SDK de Java), otimamente algo aberto e conforme os standards de \href{https://openjdk.org/}{OpenJDK}, o qual usei \href{https://docs.aws.amazon.com/corretto/latest/corretto-17-ug/downloads-list.html}{Amazon Corretto}, visto à sua licença aberta e gratuita, multiplataforma e vem com suporte de longo prazo que incluirá melhorias de desempenho e correções de segurança.

Para gestão de pacotes e tarefas, precisamos de um gestor de pacotes, o qual usei o \href{https://gradle.org/}{Gradle}, e como um dos meus computadores de trabalho usa \href{https://www.microsoft.com/pt-pt/software-download/windows10} em vez de \href{https://archlinux.org/download/}{Linux}, a instalação do \href{https://gradle.org/}{Gradle} sem um gestor de pacotes e alteração do path, dá-nos jeito usar um IDE que trate desses assuntos, o qual foi-me recomendado (e usado): \href{https://www.jetbrains.com/idea/}{IntelliJ IDEA} da \href{https://www.jetbrains.com/}{JetBrains}.

Para hospedar a base de dados e o projeto, numa pequena rede de containers interna, foi instalado o \href{https://www.docker.com/}{Docker} no \textit{desktop} \href{https://www.microsoft.com/pt-pt/software-download/windows10}{Windows} (e usado o \href{https://podman.io/}{Podman} no portátil \href{https://archlinux.org/download/}{Linux}, pelo simplesmente facto de já o ter instalado previamente).

No entanto ainda nos falta algo bastante importante. Nomeadamente, algo quer faça o Bootstrap do projeto em \href{https://micronaut.io/}{Micronaut}, para isso temos variadas opções:

\begin{itemize}
  \item Ir ao o \href{https://micronaut.io/launch}{Micronaut Launch Website}
  \item Usar o \href{https://micronaut.io/download/}{Micronaut CLI} \href{https://micronaut-projects.github.io/micronaut-starter/latest/guide/index.html}{em que temos aqui a documentação}
  \item Fazer \href{https://curl.se/}{curl} à API do \href{https://micronaut.io/}{Micronaut} Launch \href{https://launch.micronaut.io/create/default/com.optiply.project.webshop?lang=JAVA&build=GRADLE&test=JUNIT&javaVersion=JDK_17&features=jackson-databind&features=kubernetes-reactor-client&features=properties&features=flyway&features=jdbc-hikari&features=jooq&features=postgres&features=r2dbc&features=testcontainers&features=lombok&features=mockito&features=openrewrite&features=asciidoctor&features=logback&features=reactor&features=security-jwt&features=problem-json&features=jackson-xml}{https://launch.micronaut.io/create/default}
\end{itemize}

\subsection{Inicio do Projeto}

No meu caso em especifico foi-me fornecido um repositório privado no \href{https://bitbucket.org/}{BitBucket}, o qual apenas me foi necessário fazer uma \textit{fork}. O estado desse repositório e da \textit{fork} pode ser visto no neste \href{https://github.com/CatKinKitKat/MicronautJooqPostgresREST/tree/07d359ce933dde634f176dc95bf5ac1b3e4bc93d}{\textit{commit}} (\href{https://github.com/CatKinKitKat/MicronautJooqPostgresREST}{num repositório meu} do \href{https://github.com/}{GitHub}, onde no projeto o adicionei como segunda origem, para backup).

Esta diretoria de projeto nos atribuída, pessoalmente achei que era maior e mais complicada que o necessário, talvez esta seja única e o que varia são os projetos que a usam. Com isso em conta eu decidi, fazer uma redução ao projeto, para que ficasse mais simples de trabalhar e não houvessem pacotes ou funcionalidades que não fossem necessárias. Isto pode ser observado neste \href{https://github.com/CatKinKitKat/MicronautJooqPostgresREST/commit/3c71d709599662436ae13cf9dcf609a5ca5464e3}{\textit{commit}} (o qual descrição reflete o meu estado mental sobre determinada observação).

Após a redução, o projeto ficou com apenas dois subprojetos para o Gradle gerir, um que contem a aplicação em sí e o outro que trata dos repositórios/classes de transações à base de dados. O numero de pacotes externos e funcionalidades foi reduzido para o mínimo necessário, esses incluíram: \href{https://flywaydb.org/}{Flyway}, \href{http://fasterxml.com/}{Jackson}, \href{https://www.jooq.org/}{jOOQ}, \href{https://junit.org/junit5/}{JUnit}, \href{https://logback.qos.ch/}{Logback}, \href{https://projectlombok.org/}{Lombok}, \href{https://site.mockito.org/}{Mockito}, \href{https://jdbc.postgresql.org/}{Postgres}, \href{https://r2dbc.io/}{R2DBC} e \href{https://projectreactor.io/}{Reactor}.

\newpage

\subsubsection{Detalhes sobre as tecnologias}

\paragraph{\href{https://flywaydb.org/}{Flyway}\\}

O \href{https://flywaydb.org/}{Flyway} é um framework de migrações de bases de dados, que é usado para gerênciar as migrações de bases de dados de projetos Java. Funciona de maneira semelhante às migrações nativas do ASP.NET Core.

Este pacote adiciona essas capacidades a tarefas do Gradle, como o \textit{flywayMigrate} e \textit{flywayInfo}. A migrações são feitas através de um ficheiro de migrações, que é um ficheiro de SQL, dentro da diretoria de migrações (\textit{PROJECT\_ROOT/src/main/resources/db/migrations}), com a versão em que a migração deve ser executada e dois \textit{underscores}.

\paragraph{\href{http://fasterxml.com/}{Jackson}\\}

O \href{http://fasterxml.com/}{Jackson} é um framework de serialização de objetos, que é usado para serializar objetos em JSON.

A serialização é um processo de transformação de um objeto em um JSON, e a deserialização é o processo de transformação de um JSON em um objeto.

Isto é feito principalmente através de um objeto \textit{ObjectMapper}, que é um objeto que implementa a interface \textit{com.fasterxml.jackson.databind.ObjectMapper}.

\paragraph{\href{https://jooq.org/}{jOOQ}\\}

O \href{https://jooq.org/}{jOOQ} é um framework de código-fonte de código-aberto, que é usado para gerir a base de dados. Este funciona de maneira semelhante às operações do Entity Framework Core para ASP.NET Core, sendo que este abstrai as operações de SQL em wrappers programáticos.

\paragraph{\href{https://junit.org/junit5/}{JUnit}\\}

O \href{https://junit.org/junit5/}{JUnit} é um framework de testes, que é usado para gerênciar os testes de unidades. Usado muito na disciplina de Programação Orientada a Objetos.

\paragraph{\href{https://logback.qos.ch/}{Logback}\\}

O \href{https://logback.qos.ch/}{Logback} é um framework de logging, que é usado para gerir os logs de um projeto, com o foco em abstrair o uso de logs ao mais simples possível. É o sucessor do \href{https://log4j.org/}{Log4j}, que foi alvo de uma vulnerabilidade recentemente.

No Windows devemos alterar uma configuração: a desativação do JANSI, que não funciona com alguns Locales, em especial os que o Windows usa.

\paragraph{\href{https://projectlombok.org/}{Lombok}\\}

O \href{https://projectlombok.org/}{Lombok} é um framework de código-fonte de código-aberto, que é usado para gerir a criação de classes de objetos através de anotações. Com estas anotações, abstraímos o código, evitamos repetição e automatizamos muito o processo desenvolvimento.

Por exemplo a anotação \textit{@Getter} faz com que o Java crie automaticamente os getters. Ou, a anotação \textit{@Data} faz com que o Java crie automaticamente os getters, setters, equals, hashCode, toString e clone.

\paragraph{\href{https://site.mockito.org/}{Mockito}\\}

O \href{https://site.mockito.org/}{Mockito} é um framework auxiliar de testes, que é usado para gerir os mocks de objetos. Os mocks são objetos que são usados para simular o comportamento de objetos realmente existentes.

Com os mocks, podemos testar objetos que ainda não existem, como um objeto de um repositório de dados, ou um objeto de um serviço. Como também isolamos o comportamento dos objetos, evitamos que os objetos sejam alterados durante o teste ou para testar apenas o comportamento do que comunica com o mesmo.

\paragraph{\href{https://jdbc.postgresql.org/}{PostgreSQL}\\}

O \href{https://jdbc.postgresql.org/}{PostgreSQL} é um driver JDBC, que é usado para conectar a bases de dados PostgreSQL. Um driver JDBC é um driver que permite ao Java a comunicação com bases de dados.

PostgreSQL é o tipo de base de dados usado no projeto.

\paragraph{\href{https://r2dbc.io/}{R2DBC}\\}

O \href{https://r2dbc.io/}{R2DBC} é um driver de conexão à base de dados, mas contrariamente ao anterior este permite fazer transações reativas como as do Project Reactor ou do RxJava.

\paragraph{\href{https://projectreactor.io/}{Reactor}\\}

O \href{https://projectreactor.io/}{Reactor} é um framework de eventos, que é usado para gerir eventos e criar aplicações reativas. Uma aplicação reativa é uma aplicação que é executada em um fluxo de eventos.

\subsection{Paradigma de Programação}

Programação reativa é o acto de programar para trabalhar com fluxos de dados assíncronos. Isto é importante devido o crescimento da Internet e a demanda enorme de dados em tempo real. Esta programação precisa de ser dinâmica, ou seja; diferente das formas tradicionais de desenvolvimento.

Nas formas tradicionais de programar/desenvolver, de modo \textbf{muito} genérico, cria-se variadas tarefas e elas comunicam-se em tempos pré-determinados, com respostas pré-determinadas, são ``rígidas'', seguem regras diretas.

Isto funciona e continua a ser utilizada até hoje, entretanto esta ``lógica'' não é compatível com as necessidades de alguns serviços atuais e os seus inúmeros clientes e dados. Na programação reativa isto ocorre de uma forma semelhante, mas mais inteligente, interligada em paralelo, sem seguir uma ordem cronológica e linear.

\newpage

Os pilares da programação reativa são:

\begin{itemize}
  \item \textbf{Elástico:} Reage à demanda/carga: aplicações podem fazer uso de múltiplos núcleos e múltiplos servidores;
  \item \textbf{Resiliente:} Reage às falhas; aplicações reagem e se recuperam de
    falhas de software, hardware e de conectividade;
  \item \textbf{Message Driven:} Reage aos eventos (event driven): em vez de compor
    aplicações por múltiplas threads síncronas, sistemas são compostos de gerenciadores de eventos assíncronos e não bloqueantes;
  \item \textbf{Responsivo:} Reage aos usuários: aplicações que oferecem interações
    ricas e ``tempo real'' com usuários.
\end{itemize}

\subsection{Estrutura do Projeto}

A estrutura do projeto, como dito anteriormente, foi uma modificação do herdado da estrutura inicial vinda do repositório oferecido para \textit{forking}. Este projeto consitis de um projeto Gradle com três subprojetos: um pacote com classes entendidas de Monos, um pacote para o core do projeto, e um pacote para os repositórios que tratam das transações com a base de dados.

Sendo que as classes eram apenas \textit{MonoVoid}, \textit{MonoFalse} e \textit{MonoTrue} que simplesmente implementavam a interface \textit{Mono} e retornavam um valor booleano (ou nenhum), decidi cortá-las visto que não trazem quais quer nova funcionalidade ao projeto e não me custa escrever \textit{Mono<Boolean>} e retornar um valor booleano.

Com isto, a estrutura do projeto foi alterada para um projeto Gradle com dois subprojetos.

\subsection{Metodologia de desenvolvimento}

Foi-me notificado que o projeto seria desenvolvido de forma livre, sem qualquer metodologia de desenvolvimento. No entanto, sendo eu um alguém novo na area e a trabalhar remotamente, decidi que é extremamente importante arranjar um ambiente de desenvolvimento que me permita trabalhar, ponto. Por isto, decidi referir às estratégias de gestão de projeto ensinadas na disciplina de Engenharia de Software, como o \textit{Scrum} e o \textit{Kanban}.

Sabendo que a empresa onde estou usa \textit{Scrum}, ponderei usar o mesmo e as ferramentas disponíveis no \href{https://www.atlassian.com/}{Atlassian} como \textit{Jira} e \textit{Confluence}; mas acabei por decidi usar uma estratégia menos rígida, o \textit{Kanban}.

\newpage

\subsubsection{Kanban}

Para o meu projeto, a estratégia de desenvolvimento foi o \textit{Kanban}, ou pelo menos uma forma primitiva do mesmo. Expandindo, foi feito um quadro de tarefas divididas em cinco colunas: \textit{Tarefas}, \textit{Por aprender}, \textit{Por implementar} \textit{Por testar} e \textit{Terminadas}.

A metodologia de trabalho começava por ir identificando tarefas, se muito complexas dividi-las em pequenas tarefas e julgando a minha capacidade de as fazer. Colocando na coluna respetiva e depois trabalhando de acordo com o estado do quadro.

Este quadro infelizmente já não está acessível, pois a conta empresarial já foi fechada.

\subsubsection{Nota sobre a escolha}

Na minha opinião subjetiva, a escolha do \textit{Kanban} sobre \textit{Scrum}, foi uma boa decisão, visto que o \textit{Scrum} é um padrão mais rígido e linear, havendo a (extra) necessidade em criar \textit{user-stories} e em definir \textit{sprints}. Já o \textit{Kanban} é um padrão mais flexível, tanto por ter menos etapas para organizar como por ter um quadro de tarefas menos \textit{standard} e mais flexível.

\subsection{Testes}

Para testar o software, foi recomendado uma mistura de \textit{JUnit} e \textit{Mockito} ou usar \textit{Spock}. Estes testes foram feitos dentro do subprojeto principal, e foram divididos em duas partes:

\begin{itemize}
  \item \textit{Testes de unidade}
  \item \textit{Testes de integração}
\end{itemize}

Houve também uma secção chamada \textit{shared}, onde havia um conjunto de classes que orientava o pacote do \textit{TestContainers} e o seu \textit{container} de teste \textit{PostgreSQL}.

Noto que uma dos requerimentos do projeto era que um pacote do Gradle, chamado JaCoCo e que verifica a percentagem de código testado, fosse incluído no projeto e o resultado mínimo obtido fosse de 80\%. O que foi feito e entregue, com 82\% na entrega final e que houve uma altura que foi entregue com uns 100\% de coverage (na secção seguinte 4.3.6 saberão porquê), no então noto também que este relatório refere-se principalmente ao estado da entrega final.

\subsubsection{Testes de unidade}

Os testes de unidade são testes que testam uma unidade do software, isto quer dizer que testam algo em isolamento do resto do software. Este tipo de testes permitem verificar o correto funcionamento daquela especifica classe ou função, assim permitir identificar (ou excluir da procura) \textit{bugs} ou erros no software.

Foram feitos testes de unidade para as classes dos modelos (os objetos com que comunicamos) e para as classes de serviços.

\subsubsection{Testes de integração}

Os testes de integração são testes que testam o funcionamento do software como um todo. Este tipo de testes permitem verificar o correto funcionamento do software ou permitir identificar se existe algum \textit{bug} no software, sabendo também se esse erro está na integração das unidades se em conjunto com testes de unidade sem testes falhados.

Foi feita uma serie de testes de integração para a classe do \textit{controller}, que é responsável por receber os \textit{requests} à API e assim testando o funcionamento da API num todo.

\subsection{\textit{Feedback}}

O \textit{feedback} sobre o desenvolvido seria feito sob pedido via Slack ao responsável sobre os estagiários de backend, o André Figueira.

De forma concreta eu obtive dois sets de \textit{feedbacks}:

\begin{itemize}
  \item \textit{Feedback} da primeira entrega
  \item \textit{Feedback} da segunda entrega (final)
\end{itemize}

\subsubsection{\textit{Feedback} da primeira entrega}

Este foi o \textit{Feedback} mais volumoso, que combinou comentário sobre \textit{Clean Code} e os conceitos \textit{SOLID}, comentário sobre os verbos dos métodos \textit{HTTP} e comentário sobre ler as inferências do \textit{.pdf} do projeto.

Começando de trás para a frente, o primeiro comentário foi sobre os \textit{endpoints}. Ou seja, as ações que o software pode realizar, descritas no documento, inferem os endpoints que o software deve ter e não algo que apenas os satisfaça, lembrar o que foi dito no final da secção 4.2.

Seguinte, os caminhos da URI, para os simplificar o mais possível, não precisam de conter os verbos das ações que fazem, visto que sendo ações CRUD, descritas facilmente com os métodos HTTP (GET, PUT, POST, DELETE), podem e devem ser cortados ao máximo.

Exemplo: \texttt{\textbf{HTTP DELETE ->} http://localhost/remove/\{id\}} passar para\\\texttt{\textbf{HTTP DELETE ->} http://localhost/\{id\}}.

Por último, e não menos importante, os conceitos \textit{SOLID} devem ser seguidos ao máximo independentemente do que achamos que vai ser o rumo do software. Isto porquê? Eu achei que visto que não iriam haver mais do que uma interface de comunicação com o software, sendo apenas necessário um \textit{controller} que serve a comunicação REST HTTP, não seria necessário uma outra classe com o \textit{business logic}, ou seja uma classe de serviço onde \textit{controllers} iriam buscar. Isto estava errado mesmo que a lógica sobre esta decisão não estivesse muito errada. O software deve separar o \textit{business logic} do \textit{controller} o mais possível e o software deve ser extensível.

\subsubsection{\textit{Feedback} da segunda entrega (final)}

O \textit{feedback} desta entrega foi muito mais simples, visto que todos os pontos anteriores foram corrigidos, o \textit{feedback} foi simplesmente que o software estava satisfatório com o que foi pedido se bem que a documentação poderia ter sido um pouco mais extensa.
