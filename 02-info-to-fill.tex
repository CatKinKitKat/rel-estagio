% !TeX program = LuaLaTeX
% !TeX encoding = UTF-8
% !TeX spellcheck = pt_PT
% !TeX hyphenation = pt_PT

% Para preencher
\newcommand{\ESCOLA}{Escola Superior de Tecnologia e Gestão}
\newcommand{\CURSO}{Licenciatura em Engenharia Informática}

\newcommand{\TITULO}{Relatório de Estágio}
\newcommand{\SUBTITULO}{Desenvolvimento em Backend na Optiply}

\newcommand{\NOMEALUNO}{Gonçalo Candeias Amaro}
\newcommand{\LOCAL}{Beja, Portugal}

\newcommand{\ORIENTADORIPBEJAA}{Gonçalo Fontes}
% se existir segundo orientador do IPBeja, retirar o comentário da linha seguinte
%\newcommand{\ORIENTADORIPBEJAB}{Colocar o nome do(a) segundo(a) docente orientador(a), se existente}

%se for um estágio deve ser retirado o comentário da linha seguinte e indicar o orientador na entidade de acolhimento do estágio
\newcommand{\ORIENTADORENTIDADE}{Fábio Belga}

%Completar e comentar um dos seguintes dois \newcommand
%\newcommand{\DECLARACAOPROJETO}{Relatório de projeto de fim de curso apresentado na\linebreak \ESCOLA{} do Instituto Politécnico de Beja}
\newcommand{\DECLARACAO}{Relatório de estágio, realizado na Optiply, apresentado na\linebreak \ESCOLA{} do Instituto Politécnico de Beja}

% retirar comentário e preencher se existente
%\newcommand{\DEDICATORIA}{ texto a colocar }
